% chapter 5
% Last edit: 2017-4-24
\chapter[Joint Probability Distributions]{Joint Probability Distributions and Random Samples}
\section{Jointly Distributed Random Variables}
\subsection{Two Discrete Random Variables}
$X$,$Y$ are r.v's defined on $\mathcal{S}$. The joint p.m.f is defined as 
\[p(x,y)=P(X=x,Y=y)\]
Let $A$ be an event consisting of pairs of $(x,y)$. Then
\[P\left((X,Y)\in A\right) = \sum_{(X,Y)\in A}p(x,y)\]

\begin{exmp}
Insurance company. For a newcomer, he has two insurance. Home \& Cars. Deductible amount: Auto \$100, \$250; Home \$0, \$100, \$200.
\begin{center}
\begin{tabular}{c|c|ccc}
\hline  \hline
    &     &    & $Y$ &   \\
\hline
    & $p(x,y)$ & 0    & 100  & 200  \\
$X$ & 100      & 0.2  & 0.1  & 0.2  \\
    & 250      & 0.05 & 0.15 & 0.3  \\
\hline
\end{tabular}
\end{center}
An individual home-owner is randomly selected.
\begin{align*}
P(Y \geq 100)=& P(X=100, Y=100)+P(X=250, Y=100)+P(X=100, Y=200)+\\
&P(X=250, Y=200)\\
=& 0.1+0.15+0.2+0.3=0.75
\end{align*}
\end{exmp}

\begin{defn}
``Marginal" p.m.f of $X$ and $Y$, denoted by $p_X(x)$ and $p_Y(y)$ respectively, are given by
\begin{align*}
p_X(x)&=\sum_{y}p(x,y)	\\
p_Y(y)&=\sum_{x}p(x,y)	\\
p_X(x)&=P(X=x)=P(X+x,Y=\dots)+\dots
\end{align*}
\end{defn}

\begin{exmp}
In the Insurance example,
\[P(X=x)=\begin{cases}
P_X(100)=\dots=0.5\\
P_X(250)=\dots=0.5\\
\end{cases}\]
\[P(Y=y)=\begin{cases}
P_Y(0)=\dots=0.25 \\
P_Y(100)=\dots=0.25\\
P_Y(200)=\dots=0.5\\
\end{cases}\]
\end{exmp}
\begin{center}
\begin{tabular}{c|c|ccc|c}
\hline
    &     &    & $Y$ &   & \\
\hline
    & $p(x,y)$ & 0    & 100  & 200 & $p_X(x)$  \\
$X$ & 100      & 0.2  & 0.1  & 0.2 & 0.5 \\
    & 250      & 0.05 & 0.15 & 0.3 & 0.5 \\
    \hline
    & $p_Y(y)$ & 0.25 & 0.25 & 0.5 & 1 \\
\hline
\end{tabular}
\end{center}

\subsection{Two Continuous Random Variables}
$(X,Y)$ continuous r.v. $f(x,y)$ is the joint p.d.f of $X$ and $Y$ if for any 2-dimensional set.
\[P((X,Y)\in A)=\iint_{A} f(x,y)\,dx \,dy\]
Particularly for $A =\{(x,y),a \leq x\leq b,c \leq y \leq d\}$,
\begin{align*}
P(A)&=P(a\leq X \leq b,c  \leq Y \leq d)\\
&= \int_{a}^b \int_c^d f(x,y) \,dy \,dx=\int_c^d \int_{a}^b f(x,y) \,dx \,dy
\end{align*}

\begin{exmp}
\begin{align*}
X=&\text{ right front tyre pressure}\\
Y=&\text{ left front tyre pressure}
\end{align*}
\[f(x,y)=\begin{cases}
k(x^2+y^2)   & 20\leq x,y \leq 30\\
0, 		& o.w.
\end{cases}\]
(1) What is $k$?
\begin{align*}
1=& \int_{20}^{30}\int_{20}^{30} k(x^2+y^2) dx dy = k\int_{20}^{30}  \left(\left.\left( \frac{x^3}{3}+xy^2 \right)\right|_{20}^{30} \right)dy \\
=& k \int_{20}^{30}  \left(\frac{19000}{3}+10y^2\right)dy=k\left(\frac{19000}{3}+\frac{19000}{3}\right) \\
&\Rightarrow k=\frac{3}{38000}
\end{align*}
(2)\begin{align*}
P(X\leq 26, Y\leq 26)=& \int_{20}^{26}\int_{20}^{26} k(x^2+y^2) dx dy\\
=&   k \int_{20}^{26}  \left(\frac{26^3-20^3}{3}+6y^2\right)dy\\
=&	2k\cdot 6 \cdot\frac{1}{3} (26^3-20^3)=0.3024
\end{align*}
\end{exmp}

\begin{defn}
Marginal rv
\begin{align*}
f_X(x)&=\int_{-\infty}^{\infty} f(x,y) dy	\\
f_Y(y)&=\int_{-\infty}^{\infty} f(x,y) dx	
\end{align*}
\end{defn}

\begin{exmp}
Example 5.4 (continued)

(3)
\begin{align*}
f_X(x)=& \int_{20}^{30} k(x^2+y^2) dy =k  \left. \left( \frac{y^3}{3}+x^2 y \right)\right|_{20}^{30}  \\
=& k\left(10x^2 +\frac{19000}{3}\right)=\frac{3}{38000}x^2+\frac{1}{20} \qquad 20 \leq x \leq 30
\end{align*}
\[f_Y(y)=\frac{3}{38000}y^2+\frac{1}{20} \qquad 20\leq y \leq 30\]
(4)
\[P(20 \leq X \leq 25)=\int_{20}^{25} f_X(x)dx=\dots=0.45\]

\end{exmp}

\begin{defn}
Expected Values
\[E[h(X,Y)]=\int_{-\infty}^{\infty} \int_{-\infty}^{\infty} h(x,y) f(x,y) dx dy \]
\end{defn}

\begin{exmp}
Example 5.4 (continued)

(5)
\[E(X+Y)=\int_{20}^{26}\int_{20}^{26}(x+y) k(x^2+y^2) dx dy=\dots\]
\end{exmp}

\subsection{Independent Random Variables}
$X$ and $Y$ are said to be independent, if 
\begin{align*}
discrete: p(x,y)&=p_X(x)p_Y(y) \text{ for all }(x,y)\\
continuous: f(x,y)&=f_X(x)f_Y(y) \text{ for all }(x,y)
\end{align*}

\begin{exmp}
Example 5.4 (continued)

(6)
\begin{align*}
f(x,y) =& \frac{3}{38000} (x^2+y^2) \qquad 20 \leq x,y \leq 30\\
f_X(x) =&  \frac{3}{38000}x^2+\frac{1}{20} \qquad 20 \leq x \leq 30 \\
f_Y(y) =&  \frac{3}{38000}y^2+\frac{1}{20} \qquad 20 \leq y \leq 30
\end{align*}
\[f(x,y)\neq f_X(x)f_Y(y) \qquad \text{for } x=20, y=20\]
$X$ and $Y$ are not independent. 
\end{exmp}

\begin{exmp}
\[f(x,y)=\lambda_1\lambda_2 e^{-\lambda_1 x - \lambda_2 y}, x\geq0,y\geq0\]
Are $X$ and $Y$ independent?
\begin{align*}
f_X(x)&=\int_{0}^{\infty} \lambda_1 \lambda_2 e^{-\lambda_1 x-\lambda_2 y} dy \\
&=\lambda_1 e^{-\lambda_1 x} \int_{0}^{\infty} \lambda_2 e^{-\lambda_2 y} dy = \lambda_1 e^{-\lambda_1} \qquad x \geq 0
\end{align*}
\[f_Y(y)=\lambda_2 e^{-\lambda_2} \qquad y \geq 0\]
\[f(x,y)= f_X(x)f_Y(y) \qquad \text{for any } (x,y)\]
So, $X \!\perp\!\!\!\perp Y$.
\end{exmp}


\subsection{More Than Two Random Variables}
\begin{defn}
$X_1,X_2,\dots,X_n$ are discrete rvs', the joint p.m.f is defined as
\[p(x_1,x_2,\dots,x_n)=P(X_1=x_1,\dots ,X_n=x_n)\]
In continuous case, the joint p.d.f $f(x_1,x_2,\dots,x_n)$ is such that
\[P(A) =\int_A\dots\int f(x_1,x_2,\dots,x_n) d x_n \dots d x_1\]
\end{defn}

Particularly, $A=\{a_1\leq x_1 \leq b_1,\dots,a_n\leq x_n \leq b_n\}$
\begin{align*}
P(A)=&P(a_1\leq x_1 \leq b_1,\dots,a_n\leq x_n \leq b_n)\\
&=\int_{a_1}^{b_1}\dots\int_{a_n}^{b_n} f(x_1,x_2,\dots,x_n) d x_n \dots d x_1
\end{align*}

\begin{exmp}
A dice is rolled 100 times. $X_i=$\# of $i$ dots out of 100 times; $i=1,2,\dots,6$
\[p_i=P(i \text{ dots}) \qquad p_1+p_2+\dots+p_6=1\]
\[P(X_1=x_1,\dots,X_6=x_6)=\frac{100!}{x_1!x_2!\dots x_6!}p_1^{x_1}p_2^{x_2}\dots p_6^{x_6} \quad(0\leq x_1,\dots,x_6 \leq 100, x_1+\dots+x_6=100)\]
\end{exmp}

\begin{exmp}
$(X_1,X_2,X_3)$ has the joint p.d.f
\[f(x_1,x_2,x_3)=\begin{cases}
kx_1 x_2(1-x_3)   & 0\leq x_1,x_2,x_3 \leq 1,x_1+x_2+x_3\leq 1\\
0, 		& o.w.
\end{cases}\]
(1) What is $k$?
\begin{align*}
1&=\iiint kx_1 x_2(1-x_3) \,dx_3 \,dx_2 \,dx_1 \\
&=\int_0^1 \int_0^{1-x_1} \int_{0}^{1-x_2-x_1} kx_1 x_2(1-x_3) \,dx_3 \,dx_2 \,dx_1 \\
&= \frac{k}{144} \Rightarrow k=144
\end{align*}
(2)
\begin{align*}
P(X_1+X_2 \leq 0.5)=\iiint kx_1 x_2(1-x_3) \,dx_3 \,dx_2 \,dx_1 =0.606
\end{align*}
$0\leq x_1,x_2,x_3 \leq 1, x_1+x_2 \leq 0.5$ \footnote{The value might be wrong, run this in Mathematica -\texttt{ Integrate[144*x y (1 - z), \{x, 0, 0.5\}, \{y, 0, 0.5 - x\}, \{z, 0, 1 - x - y\}]} }
\end{exmp}

\subsubsection{Independence}
\begin{defn}
$X_1,X_2,\dots,X_n$ are independent if 
\[p(x_1,x_2,\dots,x_n)=p_{X_1}(x_1)\dots p_{X_n}(x_n)\]
or
\[f(x_1,x_2,\dots,x_n)=f_{X_1}(x_1)\dots f_{X_n}(x_n)\]
for all possible $(x_1,\dots,x_n)$
\end{defn}

\subsection{Conditional Distributions}

\begin{defn}
$(X,Y)$ with $f(x,y),f_X(x),f_Y(y)$, then the conditional p.d.f of $Y$ given $X=x$,
\[f_{Y|X}(y|x)=\frac{f(x,y)}{f_X(x)}\]
for discrete case
\[p_{Y|X}(y|x)=\frac{p(x,y)}{p_X(x)}\]
\end{defn}

\begin{exmp}
\[f(x,y)=\begin{cases}
\frac{6}{5} (x+y^2) & 0\leq x\leq 1,0\leq y\leq 1 \\
0 & o.w.
\end{cases}\]
\begin{align*}
f_X(x)=& \int_0^1 \frac{6}{5} (x+y^2) \,dy = \left.\left( \frac{6}{5}xy +\frac{6}{5}\frac{1}{3}y^3\right)\right|_0^1\\
=& \frac{6}{5} x +\frac{2}{5}. \qquad 0\leq x \leq 1 
\end{align*}
\[f_{Y|X}(y|0.8)=\frac{f(0.8,y)}{f_X(0.8)}=\frac{15}{17}y^2+\frac{12}{17} \qquad 0\leq y \leq 1 \]
\[E(Y|X=0.8)=\int_0^1 y f_{Y|X}(y|0.8) \,dy=\int_0^1 y\left(\frac{15}{17}y^2+\frac{12}{17}\right)\,dy=\frac{39}{68}\]
\end{exmp}

\section{Expected Values, Covariance, and Correlation}
\begin{prop}
\[E[h(x,y)]=\begin{cases}
\sum_{x}\sum_{y} h(x,y)p(x,y) & discrete \\
\int_{-\infty}^{\infty} \int_{-\infty}^{\infty} h(x,y) f(x,y) dx dy & continuous
\end{cases}\]
\end{prop}

\begin{exmp}
\begin{align*}
X &= \text{amount of almonds}\\
Y &= \text{amount of pecans}
\end{align*}
\[f(x)=
\begin{cases}
24xy & \text{if }0\leq x,y\leq1, x+y\leq1\\
0 & o.w. 
\end{cases}\]
Unit test: almonds: \$1.00; pecans: \$1.00; peanuts: \$ 0.50
\[h(X,Y)=X+1.5Y+0.5(1-X-Y)=0.5+0.5X+Y\]
\[E[h(x,y)]=\int_{0}^{1} \int_{0}^{1-y}(0.5+0.5x+y)24xy \,dx \,dy=1.10\]
\end{exmp}

\subsection{Covariance}
\begin{defn}
\[Cov(X,Y)=E\Big((X-\mu_X)(Y-\mu_Y)\Big)\]
where $\mu_X=E(X)$,$\mu_Y=E(Y)$
\[\Rightarrow Cov(X,Y)=E(XY)-E(X)E(Y)\]
\end{defn}

\begin{exmp}
In Example 5.1
\begin{center}
\begin{tabular}{|c|c|ccc|c|}
\hline
    &     &    & $Y$ &   & \\
\hline
    & $p(x,y)$ & 0    & 100  & 200 & $p_X(x)$  \\
$X$ & 100      & 0.2  & 0.1  & 0.2 & 0.5 \\
    & 250      & 0.05 & 0.15 & 0.3 & 0.5 \\
    \hline
    & $p_Y(y)$ & 0.25 & 0.25 & 0.5 & 1 \\
\hline
\end{tabular}
\end{center}
\begin{align*}
E(X)&=100\times 0.5+ 250 \times 0.5= 175 \\
E(Y)&=\dots= 125 \\
Cov(X,Y)&=\dots=1875
\end{align*}
\end{exmp}

\begin{exmp}
\[f(x)=
\begin{cases}
24xy & \text{if }0\leq x,y\leq1, x+y\leq1\\
0 & o.w. 
\end{cases}\]
\[Cov(X,Y)=E(XY)-E(X)E(Y)\]
\[f_X(x)=\int_0^{1-x}\]
Similarly,
\[f_Y(y)=12y(1-y)^2,0\leq y\leq 1\]
\begin{align*}
E(X)&=\int_0^1 x \cdot 12x(1-x)^2 dx =\frac{2}{5}\\
E(Y)&=\frac{2}{5}\\
E(XY)&=\int_0^1  \left(\int_0^{1-y} xy \cdot 24xy \,dx\right)\,dy =\frac{2}{15}\\
Cov(X,Y)&=E(XY)-E(X)-E(Y)=-\frac{2}{75}
\end{align*}
$X,Y$ are negatively related. But Covariance cannot indicate the relation is strong or weak.
\end{exmp}

\subsection{Correlation}
\begin{defn}
The correlation coefficient of $X$ and $Y$, denoted by $Corr(X, Y)$, $\rho_{X,Y}$, or just $\rho$, is defined by
\[Corr(X,Y)=\frac{Cov(X,Y)}{\sqrt{Var(X)Var(Y)}}\]
\end{defn}

Back to Example 5.21,
\[Var(X)=E(X^2)-{E(X)}^2\]
\[E(X^2)=\int_0^1 x^2 12x (1-x)^2 \,gx= 12\int_0^1 x^3 (1-x)^2 \,dx=\frac{1}{5}\]
\[Var(X)=\frac{1}{5}-\left(\frac{2}{5}\right)=\frac{1}{25}\]
Similarly, \[Var(Y)=\frac{1}{25}\]
\[Corr(X,Y)=\frac{-\frac{2}{75}}{\sqrt{\frac{1}{25}\frac{1}{25}}}=-\frac{2}{3}\]

\begin{prop}
\textbf{Fact}: for any $X$ and $Y$,
\[-1\leq Corr(X,Y)\leq 1\]
\end{prop}

\begin{prop}
If $ac >0$\footnote{$a$ and $c$ are either both positive or negative}, then
\[Corr(aX+b,cX+d)=Corr(X,Y)\]
``unit free" 
\end{prop}

\begin{prop}
\begin{enumerate}
\item If $X$ and $Y$ are independent,then
\[Corr(X,Y)=0\]
But $Corr(X,Y)=0 \not\Rightarrow$ $X$ and $Y$ are independent.
\item If $Corr(X,Y)=1$ or $-1$ if and only if $Y=aX+b$ for some $a,b$ with $a \neq 0$.
\end{enumerate}
\end{prop}

\begin{exmp}
$X$ and $Y$ are discrete r.v.'s
\[p(x,y)=\begin{cases}
\frac{1}{4} &(x,y)=(-4,1),(4,-1),(2,2),(-2,-2)\\
0 & o.w.
\end{cases}\]
\[p_X(x)=\begin{cases}
\frac{1}{4} &x=-4,-2,2,4\\
0 & o.w.
\end{cases}\]
\[p_Y(y)=\begin{cases}
\frac{1}{4} &y=-2,-1,1,2\\
0 & o.w.
\end{cases}\]
\[E(X)=\frac{1}{4}(-4+-2+2+4)=0 \qquad E(Y)=0\]
\[E(XY)=\frac{1}{4}(-4+-4+4+4)=0\]
\[Cov(X,Y)=0-0\times0=0 \qquad Corr(X,Y)=0\]
But $X  \not\!\perp\!\!\!\perp Y$.
\end{exmp}

\begin{exmp}
$X \sim N(0,1)$, $Y=X^2 \sim \chi^2 (1)$
\[E(X)=0 \qquad E(Y)=E(X^2)=(Var(X))=1\]
\[E(XY)=E(X^3)=\int_{-\infty}^{\infty}x^3 \phi(x)\,dx=0\]
\[Cov(X,Y)=E(XY)-E(X)E(Y)=0-0\times1=0\]
\[Cov(X,Y)=0 \qquad X  \not\!\perp\!\!\!\perp Y\]
\end{exmp}

\subsection{Properties (The Distribution of a Linear Combination)}
\begin{prop}
\textbf{(1)}. $X_1,X_2 \dots X_n$ are rv's. For any constant $a_1,a_2 \dots a_n$,
\begin{align*}
E(a_1 X_1+\dots+a_n X_n)&=a_1E(X_1)+\dots +a_n E(X_n)\\
Var(a_1 X_1+\dots+a_n X_n)&=\sum_{i=1}^n a_i^2 Var(X_i)+\sum_{i\neq j} a_i a_j Cor(X_i,Y_j)\\
&=\sum_{i=1}^n a_i^2 Var(X_i)+2\sum_{i=1}^{n}\sum_{j=i+1}^{n} a_i a_j Cor(X_i,Y_j)
\end{align*}
If $X_1,X_2 \dots X_n$ are independent
\[Var(a_1 X_1+\dots+a_n X_n)=\sum_{i=1}^n a_i^2 Var(X_i)\]
\[Var(X)=Cov(X,X)=E(XX)-E(X)E(X)\]

\textbf{(2)}. If $X_1,X_2 \dots X_n$ are independent, and $X_i\sim N(\mu_i,\sigma_i^2)$. Then
\[a_1 X_1+\dots+a_n X_n \sim N(\sum_{i=1}^{n} a_i \mu_i,\sum_{i=1}^{n} a_i \sigma^2)\]
Particularly, if $\mu_i=\mu, \sigma_i=\sigma ,a_i=\frac{1}{n}(i=1,\dots,n)$, then
\[\bar{X} \sim N(\mu,\frac{\sigma^2}{n})\]
\end{prop}

\section{Statistics and Their Distributions}
\begin{exmp}
Number of certificate obtained by students: 2,1,4,2,0. 

Sample mean: $\bar{x}=\frac{2+1+4+2+0}{5}=1.8$

Sample variance: $s^2=\frac{(2-1.8)^2+(1-1.8)^2+(4-1.8)^2+\dots}{4}$

Generally, $x_1,x_2,\dots,x_n$,

Sample mean:\[\bar{x}=\frac{x_1+\dots+x_n}{n}\]

Sample variance:\[s^2=\frac{\sum_{i=1}^n (x_i-\bar{x})^2}{n-1}\]
\end{exmp}


\begin{defn}
A statistic is a function of data before sampling (or before data are observed). There is an uncertainty on what value the statistic will result.

Usually, we use upper-case letter to denote statistic, and lower-case letter to denote observe values of a statistic.
\[\bar{X}= \frac{X_1+\dots+X_n}{n} \qquad S^2=\frac{\sum (X_i-\bar{X})^2}{n}\]
$\bar{x},s^2$. $T=\frac{\bar{X}}{S}$ is also a statistic.
\end{defn}


\subsection{Random Samples}
\begin{defn}
$X_1,X_2 \dots X_n$ are said to be a random sample of size $n$ if they are independent and identically distributed (i.i.d).
\end{defn}

\subsection{Deriving the Sampling Distribution of a Statistic}
\begin{exmp}
A car dealer, tune-up charge (\$40,\$45,\$50) for (4,6,8) cylinder cars. At a particular day, of all tune-up cars, (20\%, 30\%, 50\%) are (4,6,8) cylinder cars.

The pmf of revenue is

\begin{center}
\begin{tabular}{l|c|c|c}
\hline
$x$ & 40 & 45 & 50\\
\hline
$p(x)$ & 0.2 & 0.3 & 0.5 \\
\hline
\end{tabular}
\end{center}
\[E(X)=46.5 \qquad Var(X)=15.25\]
At another day, two tune-ups are done.
\begin{align*}
X_1&=\text{revenue for the 1st car}\\
X_2&=\text{revenue for the 2nd car}
\end{align*}
Then $X_1, X_2$ iid $X$ with pmf $p(x)$, $\bar{X}=\frac{X_1+X_2}{2}$.
\begin{center}
\begin{tabular}{l|l|c|c|c}
\hline
$x_1$ &$x_2$ & $p(x_1,x_2)$ & $\bar{x}$ & $s^2$\\
\hline
40 & 40 & 0.04 & 40 & 0 \\
\hline
40 & 45 & 0.06 & 42.5 & 12.5 \\
\hline
40 & 50 & 0.10 & 45 & 50 \\
\hline
45 & 40 & 0.06 & 42.5 & 12.5 \\
\hline
45 & 45 & 0.09 & 45 & 0 \\
\hline
45 & 50 & 0.15 & 47.5 & 12.5 \\
\hline
50 & 40 & 0.10 & 45 & 50 \\
\hline
50 & 45 & 0.09 & 47.5 & 12.5 \\
\hline
50 & 50 & 0.25 & 50 & 0 \\
\hline
\end{tabular}
\end{center}

Distribution of $\bar{X}$
\begin{center}
\begin{tabular}{l|c|c|c|c|c}
\hline
$\bar{x}$ & 40 & 42.5 & 45 & 47.5 & 50\\
\hline
$p_{\bar{X}}(\bar{x})$ & 0.04 & 0.12 & 0.29 & 0.3 & 0.25 \\
\hline
\end{tabular}
\end{center}

Distribution of $S^2$
\begin{center}
\begin{tabular}{l|c|c|c}
\hline
$s^2$ & 0 & 12.5 & 50\\
\hline
$p_{S^2}(s^2)$ & 0.38 & 0.42 & 0.20 \\
\hline
\end{tabular}
\end{center}
\begin{align*}
E(\bar{X})=& 46.5 \\
E(S^2)=& 15.25 = Var(X) \\
Var(S^2)=&
\end{align*}
\end{exmp}

\begin{exmp}
(Example 5.21 in the textbook)
$X_1,X_2 \overset{iid}{\sim}exp(\lambda)$
\[f(x)=\begin{cases}
\lambda e^{-\lambda x}, 	& x >0\\
0. &\text{otherwise}
\end{cases}\]
$Y=X_1+X_2$ is the statistic of interest. $f_Y(y)=?$.
\begin{align*}
F_Y(y)=&P(Y \leq y)=P(X_1+X_2\leq y) \\
=&\iint_{X_1+X_2\leq y} f(x_1,x_2) \,dx_1 \,dx_2 =\int_0^y \int_0^{y-x_2} \lambda e^{-\lambda x_1} \lambda e^{-\lambda x_2} dx_1 \, dx_2 \\
=& \int_0^y \int_0^{y-x_2} \lambda^2 e^{-\lambda(x_1+x_2)} dx_1 \, dx_2= \dots = 1-e^{-\lambda y}- \lambda y e^{-\lambda y} \qquad 0\leq y \leq \infty
\end{align*}
\[f_Y(y)=F'_Y(y)=\lambda e^{-\lambda y} -\lambda e^{-\lambda y}+\lambda^2 y e^{-\lambda y} =\lambda^2 y e^{-\lambda y} \qquad y \geq 0 \]
\[Y\sim Gamma(2,\frac{1}{\lambda})\]
\[E(Y)=\frac{2}{\lambda} \qquad Var(Y)=\frac{2}{\lambda^2}\]
\end{exmp}

\section{The Distribution of the Sample Mean}
\begin{prop}
Let $X_1, X_2, \dots , X_n$ be a random sample from a distribution with mean $\mu$ and variance $\sigma^2$. Then 
\[E(\bar{X})=\mu \qquad  Var(\bar{X})=\frac{\sigma^2}{n} \qquad s.d(\bar{X})=\frac{\sigma}{\sqrt{n}}\]
If $T=X_1+ X_2+ \dots + X_n$,
\[E(T)=n \mu \qquad  Var(T)=n\sigma^2 \qquad s.d(T)=\sqrt{n}\sigma\]
\end{prop}

\subsection{The Case of a Normal Population Distribution}
\begin{exmp}
In a previous class of MA2506, students' final exam score $\sim N(70,20^2)$. This year, the same class, 36 students.
\[\bar{X}=\text{ average score}\]
\[P(65 \leq \bar{X} \leq 75)\]
Since $X_1, X_2, \dots , X_n \overset{iid}{\sim}(70,20^2)$, 
$\bar{X}\sim N(70,\frac{20^2}{36})$
\[P(65 \leq \bar{X} \leq 75)=P\left(\frac{65-70}{20/6} \leq \frac{\bar{X}-70}{20/6} \leq \frac{75-70}{20/6}\right)=\Phi(-1.5 \leq Z \leq 1.5)=0.8664\]
\end{exmp}

\subsection{The Central Limit Theorem}
What if $X_1, X_2, \dots , X_n \overset{iid}{\sim}(\mu,\sigma^2)$? No normality.
\begin{theo}
\textbf{The Central Limit Theorem(CLT)}

Let $X_1, X_2, \dots , X_n$ be a random sample from a distribution with mean $\mu$ and variance $\sigma^2$. Then if $n$ is sufficiently large, $\bar{X}$ has approximately a normal distribution with $\mu_{\bar{X}}=\mu$ and $\sigma_{\bar{X}}=\frac{\sigma^2}{n}$, and $T$ also has approximately a normal distribution with $\mu_{T}=n \mu$, $\sigma_{T}^2=n \sigma^2$. The larger the value of $n$, the better the approximation. Usually, $n \geq 30$.
\end{theo}