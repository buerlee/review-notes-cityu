\chapter{Point Estimation}
\section{Some General Concepts of Point Estimation}
\begin{exmp}
Population $N(\mu,1)$.
\[10.2, 9.8, 9.5, 11, 13, 9\]
A "guess" of $\mu$ can be $\frac{10.2+9.8+9.5+11+13+9}{6}=10.4$
\end{exmp}

\begin{defn}
Generally, we need to estimate a parameter $\theta$ based on a sample data set $x_1,x_2,\dots,x_n$. A point estimate of $\theta$ is a suitable statistic on $X_1,X_2,\dots,X_n$.
\end{defn}

\begin{exmp}
(Example 6.1 in textbook)
% An unfinished exmp!!
\end{exmp}

\begin{exmp}
(Example 6.2 in textbook)
% An unfinished exmp!!
\end{exmp}

\subsection{Unbiased Estimators}
\begin{defn}
An estimate $\hat{\theta}$ is said to be unbiased if
\[E(\hat{\theta})=\theta\]
Otherwise $E(\hat{\theta})-\theta$ is called the bias of $\hat{\theta}$.
\end{defn}

\begin{exmp}
$X \sim Bin(n,p)$, $\hat{p}=\frac{X}{n}$
\[E(\hat{p})=E\left(\frac{X}{n}\right)=\frac{1}{n}E(X)=p\]
So $\hat{p}$ is an unbiased estimate of $p$.
\end{exmp}

\begin{exmp}
$X_1,\dots,X_n \overset{iid}{\sim} (\mu,\sigma^2)$
\[\hat{\mu}=\bar{X} \qquad \hat{\sigma^2}=\frac{1}{n-1}\sum_{i=1}^n(X_i-\bar{X})^2\]
\[E(\hat{\mu})=E(\frac{X_1+\dots+X_n}{n})=\frac{1}{n}E(X_1+\dots+X_n)\]
\begin{align*}
E(a_1 X_1+\dots+a_n X_n)&=a_1E(X_1)+\dots +a_n E(X_n)\\
Var(a_1 X_1+\dots+a_n X_n)&=\sum_{i=1}^n a_i^2 Var(X_i) \qquad \text{ if }X_1,\dots,X_n \text{ are independent  }  
\end{align*}

\begin{align*}
\Rightarrow E(\hat{\mu})&=\frac{1}{n}(E(X_1)+\dots+E(X_n))=\frac{1}{n}(\mu+\dots+\mu)=\mu \\
 E(S^2)&= E\left(\frac{\sum_{i=1}^n(X_i-\bar{X})^2}{n-1}\right)= \frac{1}{n-1} E\left(\sum_{i=1}^n (X_i^2 -2X_i \bar{X}+\bar{X}^2) \right) \\
 &= \frac{1}{n-1} E\left( \left(\sum_{i=1}^n X_i^2 \right) - n \bar{X}^2 \right) = \frac{1}{n-1} \left( \sum_{i=1}^n E ( X_i^2 ) - n E(\bar{X}^2) \right)
\end{align*}
Recall: $Var(X_i)=E(X_i^2)-(E(X_i))^2, E(X_i)=\mu, Var(X_i)=\sigma^2$ in normal distribution. $\Rightarrow E(X_i^2)=\mu^2+\sigma^2$. $Var(\bar{X})=E(\bar{X}^2)-(E(\bar{X}))^2 \Rightarrow E(\bar{X}^2)=\mu^2 +\frac{\sigma^2}{n}$.
\[E(S^2)=\frac{1}{n-1} \left( \sum_{i=1}^n (\mu^2+\sigma^2) - n \left(\mu^2 +\frac{\sigma^2}{n}\right) \right)=\sigma^2 \]
So, $S^2=\frac{1}{n-1}\sum_{i=1}^n(X_i-\bar{X})^2$ is an unbiased estimate of $\sigma^2$. If we use $\frac{1}{n}\sum_{i=1}^n(X_i-\bar{X})^2$, would generate a bias.
\end{exmp}

\begin{exmp}
(Example 6.4 in textbook)
% An unfinished exmp!!
\end{exmp}

\begin{prop}
If $X_1, X_2,\dots, X_n$ is a random sample from a distribution with mean $\mu$, then $\bar{X}$ is an unbiased estimator of $\mu$. If in addition the distribution is continuous and symmetric, then $\tilde{X}$ (Sample median) and any trimmed mean are also unbiased estimators of $\mu$.
\end{prop}

\subsection{Estimators with Minimum Variance}
\noindent\fbox{%
    \parbox{\textwidth}{%
        \textbf{Principle of Minimum Variance Unbiased Estimation} 
        
        Among all estimators of $\theta$ that are unbiased, choose the one that has minimum variance. The resulting is called the minimum variance unbiased estimator (MVUE) of $\theta$.
    }%
}

\begin{exmp}
(Example 6.6 in textbook)

% An unfinished exmp!!
\end{exmp}

\begin{theo}
$X_1,\dots,X_n \overset{iid}{\sim} N(\mu,\sigma^2)$, then $\hat{\mu}=\bar{X}$ is the MVUE of $\mu$.
\end{theo}

\section{Methods of Point Estimation}
\subsection{The Method of Moments}
\begin{defn}
$X_1,\dots,X_n$ random sample. \\
$k$-th sample moment:
\[\frac{1}{n}\sum_{i=1}^n X_i^k\]
$k$-th population moment:
\[E(X^k)\]
\end{defn}

\begin{defn}
Point estimation: use $\frac{1}{n}\sum_{i=1}^n X_i^k \to E(X^k)$
\end{defn}

\begin{exmp}
$X_1,\dots,X_n \overset{iid}{\sim} N(\mu,\sigma^2)$
\[E(X)=\mu \qquad E(X^2)=(E(X))^2+Var(X)=\mu^2+\sigma^2\]
\begin{align*}       %方程组开始
\left  \{              %方程组的左边包括大括号\{
\begin{array}{ll}     %设定列阵的格式:{lll}是三个L,表示三列的对齐方式为Left对齐
\hat{\mu}=\frac{1}{n}\sum_{i=1}^n X_i \\    %$——分隔列的标记,\\——表示换行
\hat{\mu^2}+\hat{\sigma^2}=\frac{1}{n}\sum_{i=1}^n X_i^2     %$同上
\end{array}           %方程列阵的结束
\right.  \Rightarrow            %方程组的右边无符号,利用“.“来标示
\left  \{              %方程组的左边包括大括号\{
\begin{array}{ll}     %设定列阵的格式:{lll}是三个L,表示三列的对齐方式为Left对齐
\hat{\mu}=\bar{X} \\    %$——分隔列的标记,\\——表示换行
\hat{\sigma^2}=\frac{1}{n}\sum_{i=1}^n X_i^2 -\bar{X}^2=\frac{1}{n} \sum_{i=1}^n (X_i-\bar{X})^2   %$同上
\end{array}           %方程列阵的结束
\right.
\end{align*}        %方程组结束
% An unfinished exmp!!
\end{exmp}

\begin{exmp}
$X_1,\dots,X_n \overset{iid}{\sim} Unif(0,\theta)$
\[E(X)=\frac{\theta}{2}\]
\[\frac{\hat{\theta}}{2}=\frac{1}{n}\sum_{i=1}^n X_i \Rightarrow \hat{\theta}=2\bar{X}\]
% An unfinished exmp!!
\end{exmp}

\begin{exmp}
(Example 6.13 in textbook)

$X_1,\dots,X_n \overset{iid}{\sim} Gamma(\alpha,\beta)$
\begin{align*}       %方程组开始
\left  \{              %方程组的左边包括大括号\{
\begin{array}{ll}     %设定列阵的格式:{lll}是三个L,表示三列的对齐方式为Left对齐
\hat{\alpha}\hat{\beta}=\bar{X} \\    %$——分隔列的标记,\\——表示换行
\hat{\alpha}\hat{\beta}^2+(\hat{\alpha}\hat{\beta})^2=\frac{1}{n}\sum_{i=1}^n X_i^2     %$同上
\end{array}           %方程列阵的结束
\right.  \Rightarrow            %方程组的右边无符号,利用“.“来标示
\left  \{              %方程组的左边包括大括号\{
\begin{array}{ll}     %设定列阵的格式:{lll}是三个L,表示三列的对齐方式为Left对齐
\hat{\alpha}=\frac{\bar{X}^2}{\frac{1}{n}\sum_{i=1}^n X_i^2 - \bar{X}^2} \\    %$——分隔列的标记,\\——表示换行
\hat{\beta}=\frac{1}{\bar{X}}\left( \frac{1}{n}\sum_{i=1}^n X_i^2 -\bar{X}^2 \right)  %$同上
\end{array}           %方程列阵的结束
\right.
\end{align*}        %方程组结束
% An unfinished exmp!!
\end{exmp}

\subsection{Maximum Likelihood Estimation}
\begin{exmp}
(Similar to Example 6.15 in the textbook)
A coin, $P(H)=p$, unknown.
\[X_i=\begin{cases}
1, & H\\
0. & T
\end{cases}\]
10100000001, $\hat{p}$.

The probability of the sequence happening is $p^3(1-p)^7$. try to make $p^3(1-p)^7$ large, Let $L=p^3(1-p)^7$.
\[\ln{L}=3\ln{p} +7 \ln{(1-p)}\]
\[\arg\!\max {L}=\arg\!\max{\ln{L}}\]
\[\arg\!\max{ (\log{p}+7\log{1-p}) }=\frac{3}{10}\]
\[(\log{L})'=\frac{3}{p}-\frac{7}{1-p} \qquad \hat{p}=\frac{3}{10}\]
\end{exmp}

\begin{defn}
Let $X_1,\dots,X_n$ have joint pmf or pdf $f(x_1,\dots,x_n;\theta)$. The MLE of $\theta$ is the one that maximizes the joint pdf (pmf) or $f(x_1,\dots,x_n;\hat{\theta_{MLE}})\geq f(x_1,\dots,x_n;\theta)$ for any $\theta$.
\end{defn}

\begin{exmp}
(Example 6.16 in the textbook)
\end{exmp}

\begin{exmp}
(Example 6.17 in the textbook)
\end{exmp}

\subsection{Estimating Functions of Parameters}
\noindent\fbox{%
    \parbox{\textwidth}{%
\begin{prop}
The Invariance Principle

Let $\hat{\theta_1},\dots,\hat{\theta_n}$ be the mle’s of the parameters $\theta_1,\dots,\theta_m$. Then the mle of any function $h(\theta_1,\dots,\theta_m)$ of these parameters is the function $h(\hat{\theta_1},\dots,\hat{\theta_m})$ of the mle’s.
\end{prop}
	}
}

\begin{exmp}
(Example 6.20 in the textbook)
the mle for $\sigma$ is $\sqrt{\frac{1}{n} \sum_{i=1}^n (X_i -\bar{X})^2 } $
\[h(\mu,\sigma^2)=\sqrt{\sigma^2}\]
\end{exmp}

\begin{exmp}
$X_1,\dots,X_n \overset{iid}{\sim} f(x;\theta)=\begin{cases}
(\theta+1)x^{\theta} & 0 \leq x \leq 1 \\
0 & o.w.
\end{cases}$, with $\theta>-1$
\begin{enumerate}
\item Use MM
\[E(X)=\int_0^1 x(\theta+1)x^{\theta} \,dx= (\theta+1)\left.\frac{x^{\theta+2}}{\theta+2}\right|_0^1 =\frac{\theta+1}{\theta+2}\]
\[\hat{E(X)}=\frac{1}{n}\sum_{i=1}^n X_i\]
\[\frac{\theta+1}{\theta+2}=\bar{X} \Rightarrow \hat{\theta}_{MM}=\frac{2\bar{X}-1}{1-\bar{X}}\]
\item Use MLE
\[L(\theta)=f(x_1,\dots,x_n;\theta)=\prod_{i=1}^n (\theta+1)x_i^n=(\theta+1)^n \left(\prod_{i=1}^n x_i\right)^{\theta}\]
\[l(\theta)=n\log{\theta+1}+\theta \sum_{i=1}^n \log{x_i}\]
\[l'(\theta)=\frac{n}{\theta+1}+\sum_{i=1}^n \log{x_i} \Rightarrow \hat{\theta}_{MLE}=-\frac{n}{\sum_{i=1}^n \log{X_i}}-1\]
\item Compare $\hat{\theta}_{MM}$ and $\hat{\theta}_{MLE}$ by their variance
\end{enumerate}
\end{exmp}

\subsection{Some Complications}
\begin{exmp}
(Example 6.22 in the textbook)
\end{exmp}