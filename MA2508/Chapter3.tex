\chapter{Partial Derivative}
\section{Multi-variables Function}
\subsection{N-dimensional Space}
$n$-th dimensional space can be considered as the $n$-th Cartesian product of $\mathbb{R}$ -- $\mathbb{R}^n$ which can be expressed as the $n$-th triple $(x_1,x_2.\cdots ,x_n)$, and denote the distance between any point in this space and the origin as the norm: $\|\mathbf{x}\|=\sqrt{{x_1}^2+{x_2}^2+\cdots +{x_n}^2}$
\subsection{Multi-variables Function}
A quantity can depend on series variables eq. Temperature $T$ can depend on position $x$ and time $t$:
\begin{center}
$T=T(x,t)$
\end{center}
\textbf{Definition:}\\
Let $\mathcal{D}$ be some region in the $xy$-plane. If there is a rule that assigns to each point in $\mathcal{D}(x,y)$ a unique real number $\mathbb{Z}$, this rule is called a function of there two variable $(x,y)$, denoted as:
\begin{center}
$z=f(x,y)$
\end{center}
\textbf{Associated terminologies:}\\
$\mathcal{D}$ is called the domain of the $f$\\
$\mathcal{R}$ is called the range of the $f$, which is the set of function value\\
\textbf{Note that:} the definition can be immediately extended to the function of more than three variables say:
\begin{center}
$g(x,y,z)$\\
$h(x,y,z,t)$
\end{center}

\subsection{Limit}
\textbf{Definition:}\\
We say that $f(x,y)$ has the limit $\mathcal{L}$ as $(x,y)$ tend to $(x_0,y_0)$ if the following criterion holds:
\\ \\For every $\varepsilon > 0$, there exist a $\delta >0$, such that $|f(x,y)-\mathcal{L}|<\varepsilon $, whenever: $0<\sqrt{(x-x_0)^2+(y-y_0)^2}<\delta$. \\
In this case, we write:
\begin{center}
$\displaystyle\lim_{(x,y) \to (x_0,y_0)} f(x,y) = \mathcal{L}$
\end{center}
If the limit exists, the value will not change along different paths.\\ \\
\textbf{Criteria of the nonexistence} of the limit at one point:\\
If along two different paths, the values corresponding to each path are different, therefore, the limit at the point do not exist.\\
This method can be used to judge existence.\\ \\
\textbf{Example:} Does the $\displaystyle\lim_{(x,y) \to (0,0)} \frac{xy^2}{x^2+y^4}$ exist?\\ \\Solution:\\
Along path one $y=x$
\begin{center}
$\displaystyle\lim_{(x,y) \to (0,0)} \frac{xy^2}{x^2+y^4}=\displaystyle\lim_{x \to 0} \frac{x^3}{x^2+x^4}=0=\mathcal{L}_1$
\end{center}
Along path two $y=-x$
\begin{center}
$\displaystyle\lim_{(x,y) \to (0,0)} \frac{xy^2}{x^2+y^4}=\displaystyle\lim_{x \to 0} \frac{x^3}{x^2+x^4}=0=\mathcal{L}_2$
\end{center}
Along path three $x=y^2$
\begin{center}
$\displaystyle\lim_{(x,y) \to (0,0)} \frac{xy^2}{x^2+y^4}=\displaystyle\lim_{y \to 0}\frac{y^4}{2y^4}=\frac{1}{2}=\mathcal{L}_3$
\end{center}
Therefore, we see that $\mathcal{L}_1=\mathcal{L}_2 \neq \mathcal{L}_3$, which indicates that the $\lim_{(x,y) \to (0,0)} \frac{xy^2}{x^2+y^4}$ does not exist.\\ \\
\textbf{Example:}  Prove that $\displaystyle\lim_{(x,y) \to (0,0)} \frac{3x^2y }{x^2+y^2}=0$.\\ \\

Solution: we need to find a $\delta <0$, such that
\begin{center}
$0<\sqrt{x^2+y^2}<\delta $\\
$\left|\frac{3x^2y }{x^2+y^2}-0\right|<\varepsilon$
\end{center}
After the scaling to the LHS of the second equation:

\[\left|	\frac{3x^2y }{x^2+y^2}	\right| \leq \left|\frac{3x^2y^2}{2xy}\right| \leq \frac{3}{2}|x| \leq \frac{3}{2}\sqrt{x^2+y^2}<\varepsilon \]

Comparing with the first formula, we get $\delta = \frac{2}{3}\varepsilon$\\
Therefore, the limit exist.
\subsection{Continuity}
The $f(x,y)$ is continuous at $(x_0,y_0)$:
\begin{enumerate}
\item $f(x_0,y_0)$ is defined
\item $\displaystyle\lim_{(x,y)\to (x_0,y_0)}f(x,y)=f(x_0,y_0)$
\end{enumerate}
Example: Determine whether\\
\[f(x,y)=\begin{cases}
\frac{2xy}{x^2+y^2} & \quad \text{if } (x,y)\neq (0,0)\\
0 & \quad \text{if } (x,y)= (0,0)
\end{cases}\]
is continuous at $(0,0)$ or not?\\ \\
Solution: $f(0,0)=0$ defined
\[\displaystyle\lim_{(x,y)\to (0,0)}\frac{2xy}{x^2+y^2}=\begin{cases}
\displaystyle\lim_{x\to 0}\frac{2x^2}{2x^2}=\frac{2}{3}=\mathcal{L}_1 & \quad \text{path 1 } x=y\\
\displaystyle\lim_{x\to 0}\frac{2x(-x)}{2x^2}=-\frac{2}{3}=\mathcal{L}_2 & \quad \text{path 2 } -x=y\\
\end{cases}\]
$\mathcal{L}_1\neq \mathcal{L}_2$ indicates that the limits does not exist, therefore, the function is discontinuous at this point.\\
Note: Continuity and limit can be expanded into higher dimension.

\section{Definition}
Partial derivative defined as following:
\begin{center}
$\frac{\partial f}{\partial x}=\displaystyle\lim_{\Delta x \to 0}\frac{f(x+\Delta x,y)-f(x,y)}{\Delta x}$
\end{center}
denoted as: $f_x(x,y)$ or $\frac{\partial f(x,y)}{\partial x}$ or $z_x$.
\subsection{Calculation of the partial derivatives}
To calculate $\frac{\partial f}{\partial x}$, we regard the $y$ as a constant and then take the derivative w.r.t $x$.
\subsection{Partial Derivatives Family}
For the function of $z=f(x,y)$
\begin{center}
$\frac{\partial f}{\partial x}=\displaystyle\lim_{\Delta x \to 0}\frac{f(x+\Delta x,y)-f(x,y)}{\Delta x}$\\
$\frac{\partial f}{\partial y}=\displaystyle\lim_{\Delta y \to 0}\frac{f(x,y+\Delta y)-f(x,y)}{\Delta y}$
\end{center}

\subsection{Example}
Find the two partial derivatives of the function: $f(x,y)=\sqrt{2xy^2+3y}$ and the find their value at $(3,3)$

Solution:

\[\frac{\partial f}{\partial x}=\frac{1}{2}\frac{\frac{\partial}{\partial x}(2xy^2+3y)}{\sqrt{2xy^2+3y}}=\frac{2y^2}{2\sqrt{2xy^2+3y}}\]

\[\frac{\partial f}{\partial y}=\frac{1}{2}\frac{\frac{\partial}{\partial y}(2xy^2+3y)}{\sqrt{2xy^2+3y}}=\frac{4xy+3}{2\sqrt{2xy^2+3y}}\]

\[\displaystyle\left.\frac{\partial f}{\partial x}\right|_{(3,3)}=\frac{3}{\sqrt{7}}	\qquad \displaystyle\left.\frac{\partial f}{\partial y}\right|_{(3,3)}=\frac{13}{2\sqrt{7}}\]


\section{Higher Partial Derivatives}
Partial derivatives can be easily expanded into higher order and higher dimensions.

For $g(x,y,z)$, we can compute its partial derivatives as the following formula:

\begin{center}
$\frac{\partial g(x,y,z)}{\partial x}=\displaystyle\lim_{\Delta x\to 0} \frac{g(x+\Delta x,y,z)-g(x,y,z)}{\Delta x}$\\
$\frac{\partial g(x,y,z)}{\partial y}=\displaystyle\lim_{\Delta y\to 0} \frac{g(x,y+\Delta y,z)-g(x,y,z)}{\Delta y}$\\
$\frac{\partial g(x,y,z)}{\partial z}=\displaystyle\lim_{\Delta z\to 0} \frac{g(x,y,z+\Delta z)-g(x,y,z)}{\Delta z}$
\end{center}

For given $z=f(x,y)$, higher order of the partial derivatives are given as the following:

\begin{center}
$\frac{\partial z}{\partial x}=\frac{\partial f(x,y)}{\partial x}=g(x,y)$\\
$\frac{\partial z}{\partial y}=\frac{\partial f(x,y)}{\partial y}=h(x,y)$
\end{center}

Therefore, for the second partial derivative:

\begin{center}
$\frac{\partial g(x,y)}{\partial x}=\frac{\partial}{\partial x}\frac{\partial f(x,y)}{\partial x}=\frac{\partial^2 f(x,y)}{\partial x^2}=f_{xx}(x,y)$\\
$\frac{\partial h(x,y)}{\partial y}=\frac{\partial}{\partial y}\frac{\partial f(x,y)}{\partial y}=\frac{\partial^2 f(x,y)}{\partial y^2}=f_{yy}(x,y)$\\
$\frac{\partial g(x,y)}{\partial y}=\frac{\partial}{\partial y}\frac{\partial f(x,y)}{\partial x}=\frac{\partial^2 f(x,y)}{\partial x\partial y}=f_{xy}(x,y)$\\
$\frac{\partial h(x,y)}{\partial x}=\frac{\partial}{\partial x}\frac{\partial f(x,y)}{\partial y}=\frac{\partial^2 f(x,y)}{\partial y\partial x}=f_{yx}(x,y)$
\end{center}

About the last two second order partial derivatives symmetric to each other, their relationship can be determined by the Clairant's Theorem.

\section{Clairant's Theorem}
If $f(x,y)$ is defined in a region that contain $(x_0,y_0)$, then given that $f_{xy}$ and $f_{yx}$ are continuous at $(x_0,y_0)$, by this, the following formula hold true to itself:
\begin{equation}
\boxed{f_{xy}(x_0,y_0)=f_{yx}(x_0,y_0)}
\end{equation}

Noted that: In MA2508 and MA2158, if not states, we always assume the condition in the theorem are satisfied: $f_{xy}=f_{yx}$.

In general, by Clairant's Theorem, higher order partial derivatives can be defined as following:
\begin{center}
$\frac{\partial^{m+n}f(x,y)}{\partial x^m \partial y^n}$ or $\frac{\partial^{a+b+c}g(x,y,z)}{\partial x^a \partial y^b \partial z^c}$ or even higher.
\end{center}

Noted: For higher derivatives, computing order is important.

Example: Given
\[f(x,y)=(1+xy)\ln (1+x^2),\]

Find
\begin{center}
$\frac{\partial^{8}f(x,y)}{\partial x^5 \partial y^3}$
\end{center}

Solution:
\[\begin{cases}
\frac{\partial^{8}f(x,y)}{\partial x^5 \partial y^3}=\frac{\partial^{7}f(x,y)}{\partial x^4 \partial y^3}\frac{\partial f(x,y)}{\partial x}=\frac{\partial^{7}f(x,y)}{\partial x^4 \partial y^3}(y\ln (1+x^3)+(1+xy)\frac{2y}{1+x^2})=\cdots \\
\frac{\partial^{8}f(x,y)}{\partial x^5 \partial y^3}=\frac{\partial^{7}f(x,y)}{\partial x^5 \partial y^2}\frac{\partial f(x,y)}{\partial y}=\frac{\partial^{7}f(x,y)}{\partial x^5 \partial y^2}(x\ln (1+x^2))=0 \\
\end{cases}\]

\section{Geometrical Representation}
\subsection{Tangent Line}
The tangent line is given by the tangent vector on the surface $z=f(x,y)$ at the point $(x_0,y_0)$. However, there are infinite number of tangent line at one point that consist a tangent plane. So firstly, let's consider the tangent vector along the fixed $y$ section curve of the surface, the slope of this curve is given by the following:

\begin{center}
$k_y=\frac{\partial f}{\partial x}|_{(x_0,y_0)}$\\
$k_x=\frac{\partial f}{\partial y}|_{(x_0,y_0)}$
\end{center}

Therefore, we can express the two tangent line of the surface as following;

\begin{center}
$l_{y}: \quad z-z_0=\frac{\partial f}{\partial x}|_{(x_0,y_0)}(x-x_0)$\\
$l_{x}: \quad z-z_0=\frac{\partial f}{\partial y}|_{(x_0,y_0)}(y-y_0)$
\end{center}

Furthermore, if we know the two directional tangent vector, we can easily compute the tangent plane.
\subsection{Tangent Plane}
Given the two vectors of the above two directional vector, we can express the tangent plane as following: \\

For surface $z=f(x,y)$
\begin{equation}
\boxed{z-f(x_0,y_0)=\frac{\partial f}{\partial x}|_{(x_0,y_0)}(x-x_0)+\frac{\partial f}{\partial y}|_{(x_0,y_0)}(y-y_0)}
\end{equation}

For the surface with implicit function $F(x,y,z)=0$, the tangent plane:

\begin{equation}
\boxed{\frac{\partial F}{\partial x}|_{(x_0,y_0,z_0)}(x-x_0)+\frac{\partial F}{\partial y}|_{(x_0,y_0,z_0)}(y-y_0)\frac{\partial F}{\partial z}|_{(x_0,y_0,z_0)}(z-z_0)=0}
\end{equation}

Example: Find the equation for the tangent plane of the surface $z=\sin (x+y)$ at the point $(1,-1,0)$

Solution:
\begin{center}
$\left. \frac{\partial z}{\partial x} \right|_{(1,-1,0)}=1 \qquad \left. \frac{\partial z}{\partial y} \right|_{(1,-1,0)}=1$\\
$z=(x-1)+(y+1)=x+y$
\end{center}

\section{Partial differential}
\subsection{Recall}
If $y=f(x)$ is differentiable, that is $f(a+dx)=f(a)+\Delta y=f(a)+f'(a)dx+\varepsilon dx$, where $dx$ is the increment with $\varepsilon \to 0$ as $dx \to 0$, then $dy|_{a}=f'(a)dx$ is called differential.\\ \\

We can see that:

\begin{center}
$f(a+dx)-f(a)\approx dy|_{a}$ for small $dx$\\
$f(a+dx)\approx f(a)+dy|_{a}$
\end{center}

The value at $a+dx$ can be approximately determined.
\subsection{Definition}
$z=f(x,y)$ is said to be differentiable at $(x_0,y_0)$ if:

\begin{center}
$dz|_{(x_0,y_0)}=f(x+dx,y+dy)-f(x,y)=f_x(x_0,y_0)dx+f_y(x_0,y_0)+\varepsilon_1dx+\varepsilon_2dy$
\end{center}

where $(\varepsilon_1,\varepsilon_2)\to (0,0)$ when $(dx,dy)\to (0,0)$.\\

We call $dz|_{(x_0,y_0)}=f(x+dx,y+dy)-f(x,y)=f_x(x_0,y_0)dx+f_y(x_0,y_0)$ to be the differential of $f$ at $(x_0,y_0)$
\subsection{Linear Approximation}
\begin{center}
$f(x+dx,y+dy)\approx f(x_0,y_0)+f_x(x_0,y_0)dx+f_y(x_0,y_0)=f(x_0,y_0)+dz|_{(x_0,y_0)}$\\
$dz|_{(x_0,y_0)}=\left. \frac{\partial f}{\partial x} \right|_{(x_0,y_0)}dx + 	\left.\frac{\partial f}{\partial y} \right|_{(x_0,y_0)}dy$
\end{center}
\subsection{Criteria for Differential}
If $f_x$ and $f_y$ exist in a neighborhood of $(x_0,y_0)$ and continuous at $(x_0,y_0)$, then $f(x,y)$ is differentiable at $(x_0,y_0)$.

\section{The Chain Rule}
\subsection{Recall and Analogy}
$y=f(x)$, $x=g(t)$, the derivative of $y$ w.r.t\footnote{with respect to --- By the editor} $t$ is following:

%\begin{center}
\[\frac{dy}{dt}=\frac{dy}{dx}\frac{dx}{dt}.\]
%\end{center}
Therefore, for differentiable function $z=f(x,y)$, where $x=g(t)$ and $y=h(t)$, we have the following formula:

%\begin{center}
\[\frac{dz}{dt}=\frac{\partial f}{\partial x}\frac{dx}{dt}+\frac{\partial f}{\partial y}\frac{dy}{dt}\]
%\end{center}

\subsection{Proof}
An increment $\Delta t$ leads to the increment $\Delta x$, $\Delta y$ and $\Delta z$.

\begin{align*}
\frac{dz}{dt}=\displaystyle\lim_{\Delta t\to 0}\frac{\Delta z}{\Delta t}= & \displaystyle\lim_{\Delta t\to 0}\frac{1}{\Delta t}(f_x(x,y)dx+f_y(x,y)+\varepsilon_1dx+\varepsilon_2dy)		\\
= & \displaystyle\lim_{\Delta t\to 0}\frac{\Delta x}{\Delta t}f_x(x,y)+\displaystyle\lim_{\Delta t\to 0}\frac{\Delta y}{\Delta t}f_y(x,y)+ \lim_{\Delta t\to 0}\varepsilon_1\lim_{\Delta t\to 0}\frac{\Delta x}{\Delta t}+\displaystyle\lim_{\Delta t\to 0}\varepsilon_2\lim_{\Delta t\to 0}\frac{\Delta y}{\Delta t}		\\
= & f_x(x,y)\frac{dx}{dt}+f_y(x,y)\frac{dy}{dt}=\frac{\partial f}{\partial x}\frac{dx}{dt}+\frac{\partial f}{\partial y}\frac{dy}{dt}
\end{align*}

\subsection{Tree diagram}
\textsc{Procedure for writing down the chain formula}:

To calculate the derivative of $z$ w.r.t $t$, we can draw the tree diagram and count how many different path from $z$ to $t$. For each path, calculate the product of derivative w.r.t the local variable. Finally, add all the product together.

\begin{center}
\begin{forest}
  for tree={
    fit=band,% spaces the tree out a little to avoid collisions
  }
  [z[x[t]][y[t]]]
\end{forest}\\
\[\frac{dz}{dt}=\frac{\partial f}{\partial x}\frac{dx}{dt}+\frac{\partial f}{\partial y}\frac{dy}{dt}\]
\end{center}

Example: Given $xy+yz+zx=u$, where $x=st$, $y=\exp{st}$ and $z=t^2$, find $\frac{dy}{ds}$\\ \\
Solution:
\begin{center}
\begin{forest}
  for tree={
    fit=band,% spaces the tree out a little to avoid collisions
  }
  [u[x[s][t]][y[s][t]][z[t]]]
\end{forest}\\
$\frac{du}{ds}=\frac{\partial u}{\partial x}\frac{dx}{ds}+\frac{\partial u}{\partial y}\frac{dy}{ds}=(y+z)t+(x+z)t\exp{st}$
\end{center}

Example: Given: $z=\frac{x}{y}$, $x=r\exp{st}$, $y=rs\exp{t}$, find $\frac{\partial^2 z}{\partial t\partial r}$\\ \\
Solution:
\begin{center}
\begin{forest}
  for tree={
    fit=band,% spaces the tree out a little to avoid collisions
  }
  [z[x[r][s][t]][y[r][s][t]]]
\end{forest}\\
$\frac{\partial z}{\partial t}=\frac{\partial z}{\partial x}\frac{\partial x}{\partial t}+\frac{\partial z}{\partial y}\frac{\partial y}{\partial t}=\frac{1}{y}rs\exp{st}+(-\frac{x}{y^2}rs\exp{st})$
\end{center}

Then draw the second diagram:
\begin{center}
\begin{forest}
  for tree={
    fit=band,% spaces the tree out a little to avoid collisions
  }
  [$\frac{\partial z}{\partial t}$ [x[r][s][t]][y[r][s][t]][r][s][t]]
\end{forest}\\
$\frac{\partial^2 z}{\partial t\partial r}=\frac{\partial }{\partial r}\frac{\partial z}{\partial t}=\frac{\partial }{\partial x}(\frac{\partial z}{\partial t})\frac{\partial x}{\partial r}+\frac{\partial }{\partial y}(\frac{\partial z}{\partial t})\frac{\partial y}{\partial r}+\frac{\partial }{\partial r}(\frac{\partial z}{\partial t})=\cdots $
\end{center}

\textbf{Example: (Important)} 
If $z=f(x,y)$, where $x=r\cos \theta $ and $y=r\sin \theta $. Show that:
\[\left( \frac{\partial z}{\partial x} \right)^2 + \left( \frac{\partial z}{\partial x} \right)^2
=\left(\frac{\partial z}{\partial r}\right)^2+\frac{1}{r^2}\left( \frac{\partial z}{\partial \theta }\right)^2\]

Solution:
\begin{center}
\begin{forest}
  for tree={
    fit=band,% spaces the tree out a little to avoid collisions
  }
  [$z$[$x$[$r$][$\theta $]][$y$[$r$][$\theta $]]]
\end{forest}
\end{center}

Path 1: from the R.H.S to the L.H.S
\begin{equation}
\frac{\partial z}{\partial r}=\frac{\partial z}{\partial x}\frac{\partial x}{\partial r}+\frac{\partial z}{\partial y}\frac{\partial y}{\partial r}=\cos\theta \frac{\partial z}{\partial x}+\sin\theta \frac{\partial z}{\partial y}
\label{eq:tree1.1}
\end{equation}
\begin{equation}
\frac{\partial z}{\partial \theta}=\frac{\partial z}{\partial x}\frac{\partial x}{\partial \theta}+\frac{\partial z}{\partial y}\frac{\partial y}{\partial \theta}=-r\sin\theta \frac{\partial z}{\partial x}+r\cos\theta \frac{\partial z}{\partial y}
\label{eq:tree1.2}
\end{equation}
\begin{align}
R.H.S=(\frac{\partial z}{\partial r})^2+\frac{1}{r^2}(\frac{\partial z}{\partial \theta })^2
=&(\cos\theta \frac{\partial z}{\partial x}+\sin\theta \frac{\partial z}{\partial y})^2+\frac{1}{r^2}(-r\sin\theta \frac{\partial z}{\partial x}+r\cos\theta \frac{\partial z}{\partial y})^2	\\
\nonumber =&(\frac{\partial z}{\partial x})^2\cos^2\theta+2\frac{\partial z}{\partial y}\cos\theta \sin\theta +(\frac{\partial z}{\partial x})^2\sin^2\theta +\\
\nonumber &(\frac{\partial z}{\partial x})^2\sin^2\theta-2\frac{\partial z}{\partial y}\cos\theta \sin\theta +(\frac{\partial z}{\partial x})^2\cos^2\theta 
=L.H.S
\end{align}
\begin{center}
%$$ (Eq1)\\
%$$ (Eq2)\\
%$$
\end{center}

Path 2: from the L.H.S to the R.H.S

$(\ref{eq:tree1.1})\cdot r\cos \theta - (\ref{eq:tree1.2})\cdot r\sin \theta$, and
$(\ref{eq:tree1.1})\cdot r\sin \theta + (\ref{eq:tree1.2})\cdot r\cos \theta$
yield:
\begin{equation}
\frac{\partial z}{\partial x}=\frac{\partial z}{\partial r}\cos\theta -\frac{\partial z}{\partial \theta}\frac{\sin\theta}{r}
\end{equation}
\begin{equation}
\frac{\partial z}{\partial y}=\frac{\partial z}{\partial r}\sin\theta +\frac{\partial z}{\partial \theta}\frac{\cos\theta}{r}
\end{equation}
\[
L.H.S=(\frac{\partial z}{\partial x})^2+(\frac{\partial z}{\partial x})^2=\cdots=R.H.S
\]
%\begin{center}
%%$$\\
%%$$\\
%$$
%\end{center}

\section{Implicit Differentiation}
Given: $F(x,y)=0$, and say $y$ can be regarded as an implicit function of $x$. Find the $\frac{\mathrm{d}y}{\mathrm{d}x}$

\begin{center}
\begin{forest}
  for tree={
    fit=band,% spaces the tree out a little to avoid collisions
  }
  [$F$[$x$][$y$[$x$]]]
\end{forest}
\end{center}

Differentiating w.r.t $x$:

\begin{center}
$\frac{\partial F}{\partial x}+\frac{\partial F}{\partial y}\frac{dy}{dx}=0$\\
$\implies \frac{dy}{dx}=-\frac{\frac{\partial F}{\partial x}}{\frac{\partial F}{\partial y}}$
\end{center}

Given: $F(x,y,z)=0$, and say $z$ can be regarded as an implicit function of $x$ and $y$. Find the $\frac{\partial z}{\partial x}$ and $\frac{\partial z}{\partial x}$

\begin{center}
\begin{forest}
  for tree={
    fit=band,% spaces the tree out a little to avoid collisions
  }
  [$F$[x][y][z[x][y]]]
\end{forest}
\end{center}

Differentiating w.r.t $x$:

\begin{center}
$\frac{\partial F}{\partial x}+\frac{\partial F}{\partial z}\frac{\partial z}{\partial x}=0$\\
$\implies \frac{\partial z}{\partial x}=-\frac{\frac{\partial F}{\partial x}}{\frac{\partial F}{\partial z}}$
\end{center}

Similarly, Differentiating w.r.t $y$:

\begin{center}
$\frac{\partial F}{\partial y}+\frac{\partial F}{\partial z}\frac{\partial z}{\partial y}=0$\\
$\implies \frac{\partial z}{\partial y}=-\frac{\frac{\partial F}{\partial y}}{\frac{\partial F}{\partial z}}$
\end{center}

\section{Directional Derivatives and Gradient Vector}
Consider $z=f(x,y)$:

\begin{center}
$f_x(x_0,y_0)=\displaystyle\lim_{h\to 0}\frac{f(x_0+h,y_0)-f(x_0,y_0)}{h}$\\ (net change along the $x$-direction)\\
$f_x(x_0,y_0)=\displaystyle\lim_{h\to 0}\frac{f(x_0,y_0+h)-f(x_0,y_0)}{h}$\\ (net change along the $y$-direction)
\end{center}

Therefore, naturally, what is the net change in any arbitrary direction $\mathbf{\hat{u}}=a\mathbf{\hat{i}}+b\mathbf{\hat{j}}$

\begin{center}
$x_1=x_0+h\cos\theta \qquad a=\mathbf{\hat{u}}\cdot \mathbf{\hat{i}}$\\
$y_1=y_0+h\sin\theta \qquad b=\mathbf{\hat{u}}\cdot \mathbf{\hat{j}}$\\
$f_u(x_0,y_0)=\displaystyle\lim_{h\to 0}\frac{f(x_0+ah,y_0+bh)-f(x_0,y_0)}{h}$
\end{center}

If the limit exists, we call it to be the directional derivatives of this function $f(x,y)$ at $(x_0,y_0)$ along $\mathbf{\hat{u}}$-direction and denoted it by $D_{\mathbf{\hat{u}}}f(x_0,y_0)$

\subsection{Theorem}
If $f$ is differentiable function of $x$ and $y$

\begin{center}
$D_{\mathbf{\hat{u}}}f(x_0,y_0)=f_x(x_0,y_0)a+f_y(x_0,y_0)b \qquad \mathbf{\hat{u}}=a\mathbf{\hat{i}}+b\mathbf{\hat{j}}$
\end{center}

\begin{proof}
define $f(x_0+ha,y_0+hb)=g(h)$.
Then $g(0)=f(x_0,y_0)$, by definition:
\[
D_{\mathbf{\hat{u}}}f(x_0,y_0)=\displaystyle\lim_{h\to 0}\frac{g(h)-g(0)}{h}=\left.\frac{dg}{dh}\right|_{h=0}
\]
%\begin{center}
%$$
%\end{center}
Let $x=x_0+ah$ and $y=y_0+bh$ and differentiating w.r.t $h$:

\begin{center}
\begin{forest}
  for tree={
    fit=band,% spaces the tree out a little to avoid collisions
  }
  [$g$[x[h]][y[h]]]
\end{forest}
\end{center}
\begin{align}
\frac{dg}{dh}|_{h=0}
=&\frac{\partial f}{\partial x}|_{(x_0,y_0)}\frac{dx}{dh}|_{h=0}+\frac{\partial f}{\partial y}|_{(x_0,y_0)}\frac{dy}{dh}|_{h=0}\\
=&f_x(x_0,y_0)a+f_y(x_0,y_0)b
\end{align}
%\begin{center}
%$$\\
%$$
%\end{center}
\end{proof}
%
%Proof: 

\textbf{Observation}:
We can write the directional derivatives as the dot product of the gradient vector and the directional vector:

\begin{center}
$D_{\mathbf{\hat{u}}}f(x_0,y_0)=\nabla f(x_0,y_0)\cdot \mathbf{\hat{u}}$\\
$\nabla f(x_0,y_0)=\frac{\partial f}{\partial x}|_{(x_0,y_0)}\mathbf{\hat{i}}+\frac{\partial f}{\partial y}|_{(x_0,y_0)}\mathbf{\hat{j}}$\\
Where the operator: $\nabla = \frac{\partial }{\partial x}\mathbf{\hat{i}}+\frac{\partial }{\partial y}\mathbf{\hat{j}}$
\end{center}

Note: This gradient can be extended to higher dimension.
\subsection{Example}
Find the directional derivative of $g(x,y)=\exp{x}\cos y$ at $(1,\frac{\pi}{6})$ in the directional vector $\mathbf{v}=\mathbf{\hat{i}}-\mathbf{\hat{j}}$:

Solution:
\begin{center}
$\mathbf{\hat{v}}=\frac{\mathbf{v}}{|\mathbf{v}|}=\frac{\mathbf{\hat{i}}-\mathbf{\hat{j}}}{\sqrt{1^2+1^2}}=\frac{\mathbf{\hat{i}}}{\sqrt{2}}-\frac{\mathbf{\hat{j}}}{\sqrt{2}}$\\
$\nabla g(x,y)=\frac{\partial g}{\partial x}\mathbf{\hat{i}}+\frac{\partial g}{\partial y}\mathbf{\hat{j}}=\exp{x}(\cos y\mathbf{\hat{i}}-\sin y\mathbf{\hat{j}})$\\
$\nabla g(1,\frac{\pi}{6})=e (\frac{\sqrt{3}\mathbf{\hat{i}}}{2}-\frac{\mathbf{\hat{j}}}{2})$
\end{center}

Then, after the dot product

\begin{center}
$\nabla g(1,\frac{\pi}{6})\cdot \mathbf{\hat{v}}=\frac{\sqrt{3}+1}{2\sqrt{2}}e$
\end{center}

\subsection{Extension}
To function of three variables (dimension)

\begin{center}
$\hat{u}=a\hat{i}+b\hat{j}+c\hat{k} \qquad w=f(x,y,z)$\\
$\implies D_{\hat{u}}f(x,y,z)=\displaystyle\lim_{h\rightarrow 0}=\frac{f(x+ha,y+hb,z+,hc)-f(x,y,z)}{h}$\\
$\nabla f(x,y,z)=\frac{\partial f}{\partial x}\hat{i}+\frac{\partial f}{\partial y}\hat{j}+\frac{\partial f}{\partial z}\hat{k}$\\
$\boxed{D_{\hat{u}}f(x,y,z)=\nabla f(x,y,z)\cdot \hat{u}}$
\end{center}

\subsection{Fastest Changing Direction}
\textbf{Theorem}: The maximum value of the $D_{\hat{u}}f(\mathbf{v})$ is the $|\nabla f(\mathbf{r})|$\\ \\

\textbf{Proof}:

\begin{center}
$D_{\hat{u}}f(\mathbf{r})=\nabla f(\mathbf{r})\cdot \hat{u}$\\
$= |\nabla f(\mathbf{r})|\cdot |\hat{u}|\cos \theta \leq |\nabla f(\mathbf{r})|$
\end{center}

This suggests: along the gradient direction, the function changes fastest.

\subsection{Geometrical Meaning of the Gradient Vector}
Consider a surface $\mathcal{S}$ given by:

\begin{center}
$F(x,y,z)=0$
\end{center}

Then, we consider an arbitrary curve $\mathcal{C}$ on $\mathcal{S}$ surface that suppose that $\mathcal{C}$: $x=x(t),y=y(t),z=z(t)$
\[
F(x(t),y(t),z(t))=0.
\]
Differentiating this equation w.r.t $t$
\begin{equation}
\frac{\partial F}{\partial x}\frac{dx}{dt}+\frac{\partial F}{\partial y}\frac{dy}{dt}+\frac{\partial F}{\partial z}\frac{dz}{dt}=0.
\end{equation}
Then, we denote:
\begin{align*}
\mathbf{r}'(t)&=\frac{dx}{dt}\mathbf{\hat{i}}+\frac{dy}{dt}\mathbf{\hat{j}}+\frac{dz}{dt}\mathbf{\hat{k}}\\
\nabla F(x,y,z)&=\frac{\partial F}{\partial x}\mathbf{\hat{i}}+\frac{\partial F}{\partial y}\mathbf{\hat{j}}+\frac{\partial F}{\partial z}\mathbf{\hat{k}}.
\end{align*}
After this, rewrite the differentiating equation:
\begin{equation}
\boxed{
\nabla F(x,y,z)\cdot \mathbf{r}'(t)=0.
}
\end{equation}
This means that: $\nabla F(x,y,z)$ is the normal vector of the tangent plane since $\nabla F(x,y,z)$ is perpendicular to any tangent vector along an arbitrary curve.

Given surface $F(x,y,z)=0$,
the normal vector gives by the following equation:
\begin{equation}
\boxed{
\nabla F(x,y,z)=\frac{\partial F}{\partial x}\mathbf{\hat{i}}+\frac{\partial F}{\partial y}\mathbf{\hat{j}}+\frac{\partial F}{\partial z}\mathbf{\hat{k}},
}
\end{equation}
substitute $(x_0,y_0,z_0)$ to determine the normal vector at point $(x_0,y_0,z_0)$.

\subsection{Example}
Find the tangent plane of the surface at $(1,0,5)$:
\begin{center}
$xe^{yz}=1$
\end{center}

Solution:

Follow thee equation,
the tangent vector can be obtained:
\begin{align*}
\nabla F(x,y,z)&=\frac{\partial F}{\partial x}\mathbf{\hat{i}} + \frac{\partial F}{\partial y}\mathbf{\hat{j}}+\frac{\partial F}{\partial z}\mathbf{\hat{k}}	\\
&= e^{yz}\mathbf{\hat{i}}+xze^{yz}\mathbf{\hat{j}}+xye^{yz}\mathbf{\hat{k}}\\
&= \mathbf{\hat{i}}+5\mathbf{\hat{j}}+0\mathbf{\hat{k}}
\end{align*}
Therefore, the equation of tangent plane:
\[(x-1)+5(y-0)=0\]

\section{Maximum and Minimum Values}
Definition: (For) $z=f(x,y)$ defined in the a region $\mathcal{D}$ in the $xy$-plane)

$f(x,y)$ has local maximum (minimum) at $(a,b)$ if:

\begin{center}
$f(x,y)\leq f(a,b)$
\end{center}

for all points $(x,y)$ in the neighbourhood of $(a,b)$.\\

If this relation holds true to all points in the region $\mathcal{D}$, then $f(x,y)$ has an absolute maximum (minimum) at $(a,b)$.

\subsection{Theorem}
If f has a total extremum at $(a,b)$, the first order partial derivatives exist at $(a,b)$, then:

\begin{center}
$f_x(a,b)=f_y(a,b)=0$
\end{center}

\subsection{Critical Point for Extremum}
\begin{itemize}
\item point that satisfy both $f_x=0$ and $f_y=0$.
\item one or both partial derivatives do not exist.
\end{itemize}

local extremums must occur at those point, which one called critical points.

\subsection{Example}
Find all the critical point of this function:
\begin{center}
$f(x,y)=y\sqrt{x}-y^2-x+6y\qquad (x\geq 0)$
\end{center}

Solution:

\begin{align*}
f_x &=y\frac{1}{2\sqrt{x}}-1\\
f_y &=\sqrt{x}-2y+6
\end{align*}

From those formula, we can see that:

$f_x$ doesn't exist at $x=0$,thus ant point lay on the $y$-axis is critical point.

Then:

\begin{align*}
f_x &=y\frac{1}{2\sqrt{x}}-1=0\\
f_y &=\sqrt{x}-2y+6=0
\end{align*}

this gives another critical point $(4,4)$

\section{Second Derivative Test}
Suppose that the second derivatives of $f(x,y)$ are continuous in a disc centered at $(a,b)$, $f_x(a,b)=f_y(a,b)=0$\\

let:
\begin{center}
$\boxed{\mathcal{D}=
\begin{vmatrix}
f_{xx}(a,b) & f_{xy}(a,b)\\
f_{xy}(a,b) & f_{yy}(a,b)
\end{vmatrix}
}$
\end{center}

Then the second derivative test gives:

\begin{itemize}
\item $\mathcal{D}>0$ $\&$ $f_{xx}(a,b)>0$ $\rightarrow$ Local minimum\\
\item $\mathcal{D}>0$ $\&$ $f_{xx}(a,b)<0$ $\rightarrow$ Local maximum\\
\item $\mathcal{D}<0$ $\rightarrow$ Sudden point \\
\item $\mathcal{D}=0$ $\rightarrow$ Failure
\end{itemize}

\subsection{Absolute Test}
For the absolute extremum, we have not only carry out the second derivative test but also the boundary test.\\
For $y=f(x,y)$ defined in a region $\mathcal{D}$, then $f(x,y)$ has the absolute maximum or absolute minimum at either critical point or some point on the boundary of $\mathcal{D}$.